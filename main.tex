\documentclass{beamer}
%
% Choose how your presentation looks.
%
% For more themes, color themes and font themes, see:
% http://deic.uab.es/~iblanes/beamer_gallery/index_by_theme.html
%
\mode<presentation>
{
  \usetheme{default}      % or try Darmstadt, Madrid, Warsaw, ...
  \usecolortheme{default} % or try albatross, beaver, crane, ...
  \usefonttheme{default}  % or try serif, structurebold, ...
  \setbeamertemplate{navigation symbols}{}
  \setbeamertemplate{caption}[numbered]
} 
\setbeamercolor{upper separation line head}{bg=pink}

\defbeamertemplate{headline}{my header}{%
\vskip1pt%
\makebox[0pt][l]{\,\insertsection}%
\hspace*{\fill}\hspace*{\fill}%
\llap{\color{grey}{\normalsize\insertpagenumber\,/\,\normalsize\insertpresentationendpage\,}}

 
 \begin{beamercolorbox}[colsep=1.5pt,ht=.75ex]{upper separation line head}
 \end{beamercolorbox}
}
\setbeamertemplate{headline}[my header]
\definecolor{light sky blue}{HTML}{58FAF4}
\definecolor{lightred}{HTML}{FE2E2E}
\definecolor{lightpurple}{HTML}{A9BCF5}
%\definecolor{grey}{HTML}{424242}
\definecolor{grey}{RGB}{61,61,70}
%\setbeamercolor{title}{bg=light sky blue,fg=black}
%\setbeamercolor{title}{bg=green,fg=black}
%\setbeamercolor{frametitle}{bg=green,fg=black}
\setbeamercolor{title}{bg=white,fg=grey}
\setbeamercolor{frametitle}{bg=grey,fg=white}
\setbeamerfont{title}{family=\sffamily,size=\huge}
\setbeamerfont{subtitle}{size=\Large}
\setbeamerfont{date}{size=\large}
\setbeamerfont{author}{size=\large}
\setbeamerfont{institute}{size=\large}
\setbeamerfont{section title}{size=\huge}
								
 
\setbeamercolor{normal text}{fg=grey}
\setbeamercolor{structure}{fg=grey}
\setbeamercolor{palette primary}{fg=grey}

 \defbeamertemplate*{title page}{customized}[1][]
{
  \begin{center}
  \begin{beamercolorbox}[sep=8pt,center]{title}
      \usebeamerfont{title}\inserttitle
  \end{beamercolorbox}
  \usebeamerfont{subtitle}\usebeamercolor[fg]{subtitle}\insertsubtitle\par
  
  \bigskip
  \bigskip
  \bigskip
  
  \usebeamerfont{author}\insertauthor\par
  \vskip3em
  \usebeamerfont{institute}\insertinstitute\par
  \bigskip
  \usebeamerfont{date}\insertdate\par
  \end{center}
 % \usebeamercolor[fg]{titlegraphic}\inserttitlegraphic
}

\makeatletter
\setbeamertemplate{footline}
{
    \leavevmode%
    \hbox{%    
        \begin{beamercolorbox}[wd=.333333\paperwidth,ht=2.25ex,dp=1ex,center]{title in head/foot}%
       
        \end{beamercolorbox}%
        \begin{beamercolorbox}[wd=.333333\paperwidth,ht=2.25ex,dp=1ex,center]{title in head/foot}%
           % \usebeamerfont{title in head/foot}\insertshorttitle
           vtx meeting 
        \end{beamercolorbox}%
           \begin{beamercolorbox}[wd=.333333\paperwidth,ht=2.25ex,dp=1ex,center]{author in head/foot}%
	           \usebeamerfont{author in head/foot}\color{black}{\insertauthor}
          \end{beamercolorbox}%
   }
 }       
 \makeatother

\setbeamercolor{item}{fg=red}
\setbeamertemplate{items}[circle]


\def\sectionname{\translate{section}}
\setbeamertemplate{section page}
{
  \begin{centering}
    \begin{beamercolorbox}[sep=4pt,center]{part title}
      \usebeamerfont{section title}\insertsection\par
    \end{beamercolorbox}
  \end{centering}
}
\def\sectionpage{\usebeamertemplate*{section page}}

\AtBeginSection[]{\frame{\sectionpage}}


\usepackage[english]{babel}
%\usepackage[utf8x]{inputenc}

\title[latex beamter template]{My Latex Beamer Template}
\author{Hidemitsu Asano}
\institute[Kyoto Univ./RIKEN]{Kyoto Univ./RIKEN}
\date{\today}


\begin{document}

{
\setbeamertemplate{footline}{} 
\begin{frame}
  \titlepage
\end{frame}
}




% Uncomment these lines for an automatically generated outline.
%\begin{frame}{Outline}
%  \tableofcontents
%\end{frame}

\section{Introduction}
\begin{frame}{Terminology}
 \begin{itemize}
  \item $S$ : A market string (wti x 100 p vs .48 1.21@1.24)
  \pause
  \item $M$ : A market 
  \begin{itemize}
  	\item $product$: a financial instrument ("wti, "brent", "goog")
    \item $month$: the month for which the financial contract expires ("jan", "x", "march")
    \item $strike1..N$: represents the strike price(s) of the financial contract
    \item $strategy$: represents the strategy type of the financial contract ("put", "call", "strad")
    \item $cross$: a hedge price for the financial contract
    \item $bid$: a bid price for the financial contract
    \item $offer$: an offer price for the financial contract
  \end{itemize}
 \end{itemize}
\end{frame}


\begin{frame}{Terminology}
 \begin{itemize}
  \item $S$ : \newline\textbf{0}: "wti", \textbf{1}: "x", \textbf{2}: "100", \textbf{3}: "p", \textbf{4}: "vs", \textbf{5}: ".48", \textbf{6}: "1.21", \textbf{7}: "1.24"
  \item $M$ : 
  \begin{itemize}
  	\item $product$: $0$, "wti"
    \item $month$: $1$, "x"
    \item $strike1$: $2$, "100"
    \item $strategy$: $3$, "p"
    \item $cross$: $5$, ".48"
    \item $bid$: $6$, "1.21"
    \item $offer$: $7$, "1.24"
  \end{itemize}
 \end{itemize}
\end{frame}


\begin{frame}{Domain Complexity}
\begin{itemize}
\item Could just map all pairs $(s,m) \in (S \times M)$ to explicitly model $P(M|S)$, but...
\pause
\item $|S|$ is large (2+ million distinct messages for crude traders alone) 
\pause
\item $|M|$ is also large, albeit less than $|S|$ 
\begin{itemize}
\item only by a couple orders of magnitude  
\item example: "z 150 call" $\equiv$ "dec 150 call"\newline
\end{itemize}
\pause
\item $P(M|S)$ is still desired, but with a more efficient representation than $O(|M||S|)$
\end{itemize}
\end{frame}

\begin{frame}{Semantic Labeling (Intuition)}
\textbf{Use domain knowledge to label each token of the string}\newline
\begin{itemize}
\pause
\item Provide $X = L(S)$ where $L(S)$ \textit{labelizes} each token
\item Design $L(S)$ such that $|X| << |S|$\newline
\pause
\item We hope that $P(M|X)$ is distributed similarly to $P(M|S)$, but in practice one instance of $X$ fans out to more possible $M$'s than $S$ does
\end{itemize}
\end{frame}


\begin{frame}{Semantic Labeling (Examples)}
\begin{itemize}
\item wti x 100 c \newline\newline \textit{becomes} \newline
\newline$PRODUCT$ $MONTH$ $NUMBER$ $PRODUCT|STRATEGY$\newline
\pause
\item brent z 50/60 ps vs .43 \newline\newline \textit{becomes} \newline
\newline$PRODUCT$ $MONTH$ $NUMBER$ $OTHER$ $NUMBER$ $STRATEGY$ $OTHER$ $NUMBER$
\end{itemize}
\end{frame}

\begin{frame}{Generalization By Labeling}
$PRODUCT$ $MONTH$ $NUMBER$ $OTHER$ $NUMBER$ $OTHER$ $NUMBER$\newline
\begin{itemize}
\pause
\item brent z 50/60 ps vs .43
\item wti x 55/60 cs vs 1.23
\item go jan 120,125 fnc cross 2.78\newline\newline
\pause
\end{itemize}
\textbf{No algorithms necessary to generalize, just need some data!}
\end{frame}

\begin{frame}{Model Details}

\begin{itemize}
\item \textbf{Current Model}:
\begin{enumerate}
 \item Retain a multinomial distribution over $M$ conditioned on each observed, labelized sequence $x = L(s)$
 \item When several markets are possible given $x$, use analytics (eg. implied premiums) to filter out unlikely markets
 \item If analytics aren't available then we can maximize the posterior distribution $P(M|X=x)$ instead
\end{enumerate}
\pause
\item \textbf{Cons}:
\begin{itemize}
\item Does not learn relationships between similar sequences. "x 10 c" and "hello x 10 c" are distinct sequences and thus create independent multinomial distributions over M
\item Fails to incorporate analytical features into the input vector- can't directly query the probability model with analytical random variables
\end{itemize}
\end{itemize}
\end{frame}

\begin{frame}{Model Alternatives}
\textbf{Vectorizing the input}:
\begin{itemize}
 \item Treat each token of the sequence $x_0, x_1, ... , x_n$ as a discrete input vector of size $n$.
 \item Outputs are also a vector, one column for each atrribute of market, each value being a position from the sequence.
  \begin{itemize}
  	\item $product$: 0
    \item $month$: 3
    \item $strike1$: 1 
    \item $strike2$: 2
    \item $strategy$: 3
    \item $cross$: 4
    \item $bid$: 5
    \item $offer$: 6 
  \end{itemize}
\item Now we can use any machine learning technique that can tolerate discrete input / output vectors
\end{itemize}
\end{frame}

\section*{Conclusions}

\begin{frame}{Conclusions}
\textbf{Use domain knowledge to simplify the learning problem}
\begin{itemize}
 \item Most algorithms don't work "out of the box" with traditional machine learning techniques 
 \pause
 \item But A good abstraction can make machine learning practically unnecessary
\end{itemize}
\textbf{Future Work}
\begin{itemize}
 \item Consider sequence learning approaches, like hidden markov models or dynamic bayesian networks
 \item Incorporate analytical features directly into the probability model
 \item Unsupervised learning (use analytics to discover reasonable markets)
\end{itemize}
\end{frame}
\end{document}
